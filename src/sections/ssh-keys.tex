\section{SSH Keys}

\begin{flushleft}
	SSH keys are used for for many different things, \textit{inter alia}
\end{flushleft}

\begin{itemize}
	\item \textbf{Remote Access:} SSH ensures encrypted remote connections for users
	      and processes.
	\item \textbf{File Transfer:} SFTP, a secure file transfer protocol managed by SSH,
	      provides a safe way to manipulate files over a network.
	\item \textbf{Port Forwarding:} By mapping a client's port to the server's remote ports,
	      SSH helps secure other network protocols, such as TCP/IP.
	\item \textbf{Tunneling:} This encapsulation technique provides secure data transfers.
	      Tunneling is useful for accessing business-sensitive data online materials from unsecured
	      networks, as it can act as a handy alternative to VPN.
	\item \textbf{Network Management:} The SSH protocol manages network infrastructure and
	      other parts of the system.
\end{itemize}

\begin{flushleft}
	The two most common SSH user authentication methods used are passwords and SSH keys.
	The clients safely send encrypted passwords to the server. However, passwords are a
	risky authentication method because their strength depends on the user’s awareness of
	what constitutes a strong password. Asymmetrically encrypted SSH public-private key
	pairs are a more secure alternative to using passwords. As of August 12th, 2021 GitHub
	made the move to permanently shut down password authentication due to security concerns.
	While you can use a token-based authentication method in its place, using a passphrase
	protected SSH key are by and large still the better alternative if security is means
	anything to you. Many hosting services such as GitHub and GitLab allow you to revoke
	an active SSH key in case of loss or security breach.
\end{flushleft}

\begin{flushleft}
	First create an SSH key with a passphrase for easier authentication\footnote{if you're on Windows,
		use the Git Bash terminal. If you don't have access to WSL, copy and paste the relevant
		sections by hand.}:
\end{flushleft}

\begin{flushleft}
	\cli{ssh-keygen -t rsa -b 4096 -C \$(eval xclip -sel clip -o)}
\end{flushleft}

\begin{flushleft}
	Next start the SSH agent:
\end{flushleft}

\begin{flushleft}
	\cli{eval \$(ssh-agent -s)}
\end{flushleft}

\begin{flushleft}
	Then add and test the SSH key:
\end{flushleft}

\begin{flushleft}
	\cli{ssh-add \home/.ssh/id\_rsa}
\end{flushleft}

\begin{flushleft}
	\cli{ssh -T git@github.com}
\end{flushleft}

\begin{flushleft}
	If you didn't run into any errors in the previous step, go ahead and copy the public key to clipboard:
\end{flushleft}

\begin{flushleft}
	\cli{cat \home/.ssh/id\_rsa.pub | xclip -sel clip}
\end{flushleft}

\begin{flushleft}
	After that open \href{https://github.com/}{GitHub} and navigate to \texttt{Settings}.
	Click on the \texttt{SSH and GPG keys} tab, add your public key and give
	it a descriptive name. Starting the SSH agent for every shell session can be
	annoying sometimes, but you can configure your terminal to automatically start
	the SSH key agent with each new shell session. On Linux, it's as easy as adding
\end{flushleft}

\begin{flushleft}
	\cli{eval \$(eval ssh-agent) > dev/null 2>\&1}
\end{flushleft}

\begin{flushleft}
	your \texttt{\home/.bashrc} file. In addition to that you also need to run
\end{flushleft}

\begin{flushleft}
	\cli{echo "AddKeysToAgent yes" >> \home/.ssh/config}
\end{flushleft}

\begin{flushleft}
	Another advantage of using SSH keys is that they make it much easier to handle
	user authentication for different accounts in a shared environment. This solution
	also works accross many different hosting services, \textit{e.g.} if you want to
	keep your private GitHub account separated from your work account:
	\textcolor{gray}{{\small\verbatiminput{files/config}}}
\end{flushleft}
