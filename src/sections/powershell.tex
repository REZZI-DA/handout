\section{PowerShell}\label{powershell}

\begin{flushleft}
	\emph{In PowerShell, administrative tasks are generally performed by Cmdlets
		(pronounced command-lets), which are specialized .NET classes implementing a
		particular operation. While bash scripts primarily consume strings and text files,
		PowerShell is centered around the idea of processing and communicating through
		Objects and APIs. It easily integrates into Window's COM interfaces and the .NET
		Framework and is often regarded as the better choice for automating tasks on
		Windows operating system.}
\end{flushleft}

\subsection{Profile Setup}\label{powershell-profile-setup}

\begin{flushleft}
	There are up to six different locations for you to choose when you want to create
	a new PowerShell profile.
\end{flushleft}

\begin{flushleft}
	\cli{\$PROFIE | Get-Member -Type NoteProperty | Format-List *}
\end{flushleft}

\begin{flushleft}
	But first you will want to check if you already have a profile:
\end{flushleft}

\begin{flushleft}
	\cli{Test-Path \$HOME\textbackslash Documents\textbackslash WindowsPowerShell\textbackslash profile.ps1}
\end{flushleft}

\begin{flushleft}
	If the result of this operation is \icli{False}, go ahead and create a new profile:
\end{flushleft}

\begin{flushleft}
	\cli{New-Item -Force \$HOME\textbackslash Documents\textbackslash WindowsPowerShell\textbackslash profile.ps1}
\end{flushleft}

\begin{flushleft}
	Next set the execution policy appropriately so that you can execute your own
	PowerShell scripts:
\end{flushleft}

\begin{flushleft}
	\cli{Set-EexeutionPolicy RemoteSigned}
\end{flushleft}

\begin{flushleft}
	Now open your profile and define a \icli{\$psrc} variable so that you access
	this file more conveniently:
\end{flushleft}

\begin{flushleft}
	\cli{\$gloabl:psrc="\$HOME\textbackslash Documents\textbackslash WindowsPowerShell\textbackslash profile.ps1"}
\end{flushleft}

\subsection{Core Commands}\label{powershell-core-commands}

\begin{flushleft}
	Many PowerShell commands provide a Linux-like alias for common commands:
\end{flushleft}

\begin{flushleft}
	\cli{Get-Alias -Definition <cmdlet>}
\end{flushleft}

\begin{flushleft}
	For example, the \icli{Set-Location} Cmdlet is also accessible using any of
	these abbreviations:
	\textcolor{gray}{{\small\verbatiminput{files/cdalias.txt}}}
\end{flushleft}

\begin{flushleft}
	Create a new file. You can add the \icli{[-ItemType <file|directory>]} option to
	specify the type of file you want to create. Additionally, you can also pass
	the \icli{-Value <string>} option to initialize the file with a string value.
\end{flushleft}

\begin{flushleft}
	\cli{New-Item [-Path <path>] -Name <file>}
\end{flushleft}

\begin{flushleft}
	Move a file to a new location. This Cmdlet can also be used to rename a file:
\end{flushleft}

\begin{flushleft}
	\cli{Move-Item -Path <file> -Destination <file>}
\end{flushleft}

\begin{flushleft}
	Print the content of \icli{test.txt} to standard output:
\end{flushleft}

\begin{flushleft}
	\cli{Get-Content test.txt}
\end{flushleft}

\begin{flushleft}
	List all \icli{tex} files from the \icli{src} directory of this document,
	and format the output accordingly. Note that \icli{gci} and \icli{ft} are
	aliases for \icli{Get-ChildItem} and \icli{Format-Table}, respectively.
\end{flushleft}

\begin{flushleft}
	\cli{gci .\textbackslash src -Include *.tex -Recurse | ft Name, DirectoryName}
\end{flushleft}

\begin{flushleft}
	Remove all log files in the current working directory:
\end{flushleft}

\begin{flushleft}
	\cli{Remove-Item -Path . -Include *.log}
\end{flushleft}

\begin{flushleft}
	Remove all \icli{txt} files in a sub directory:
\end{flushleft}

\begin{flushleft}
	\cli{Get-ChildItem -Path .\textbackslash exmdir\textbackslash{} *.txt -Recurse | Remove-Item}
\end{flushleft}

\begin{flushleft}
	Remove a directory recursively. Using the \icli{-WhatIf} option enable the dry-run
	mode:
\end{flushleft}

\begin{flushleft}
	\cli{Remove-Item -Recurse -Force <dir> [-WhatIf]}
\end{flushleft}
