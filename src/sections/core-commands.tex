\section{Core Commands}\label{git-core-commands}

\begin{flushleft}
	\emph{All commands outlined in this section belong to common Git workflows.
		Simply put, they are worth memorizing because you'll find yourself typing them
		into the console frequently before you even know it.}
\end{flushleft}

\begin{flushleft}
	Initialize a new repository:
\end{flushleft}

\begin{flushleft}
	\cli{git init [path]}
\end{flushleft}

\begin{flushleft}
	Clone a repository:
\end{flushleft}

\begin{flushleft}
	\cli{git clone <url> [path] [\flag{mirror}]}
\end{flushleft}

\begin{flushleft}
	Stage a file, or all at once:
\end{flushleft}

\begin{flushleft}
	\cli{git add <path|\flag{all}>}
\end{flushleft}

\begin{flushleft}
	Commit all files in the staging area; use the \icli{-m} flag for short commit messages.
	Git will use \icli{core.editor} to determine with which editor you want to write your
	multiline commit message.
\end{flushleft}

\begin{flushleft}
	\cli{git commit [-m <msg>]}
\end{flushleft}

\begin{flushleft}
	Rename a file:
\end{flushleft}

\begin{flushleft}
	\cli{git mv <oldpath> <newnpath>}
\end{flushleft}

\begin{flushleft}
	Remove a file:
\end{flushleft}

\begin{flushleft}
	\cli{git rm <path>}
\end{flushleft}

\begin{flushleft}
	Stage a new file and append it to the last commit. Useful if you forgot to
	change something. Note that you have to force push this commit if it was
	already pushed previously. As a rule of thumb, you should never force push
	anything unless you're aware of all the negative side effects this option
	implies.
\end{flushleft}

\begin{flushleft}
	\cli{git add <file>}
\end{flushleft}
\vspace{-0.4cm}
\begin{flushleft}
	\cli{git commit \flag{amend} \flag{no-edit}}
\end{flushleft}

\begin{flushleft}
	Edit the last commit message. Because this action also irreversibly changes
	Git's history retrospectively, the aforementioned warning also applies here.
\end{flushleft}

\begin{flushleft}
	\cli{git commit \flag{amend}}
\end{flushleft}

\begin{flushleft}
	Push your commits to a remote repository:
\end{flushleft}

\begin{flushleft}
	\cli{git push [\flag{force}]}
\end{flushleft}

\begin{flushleft}
	Create a new branch based off the current branch:
\end{flushleft}

\begin{flushleft}
	\cli{git checkout -b <branch>}
\end{flushleft}

\begin{flushleft}
	Push the current branch and set the remote as upstream. An upstream is an another
	branch name, usually a remote-tracking branch, associated with a local branch.
\end{flushleft}

\begin{flushleft}
	\cli{git push \flag{set-upstream} origin <branch>}
\end{flushleft}
