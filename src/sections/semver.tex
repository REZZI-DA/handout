\section{Semantic Versioning}

\begin{flushleft}
	\emph{The section has been adapted from the website \url{https://semver.org/}
		which formally defines a widely used standard for versioning releases. While there
		exists many other versioning schemes (TeX uses an idiosyncratic version numbering
		system while Microsoft Office build numbers use a date of release format), semantic
		versioning proposes a sequence-based identification scheme with an heavy emphasize on
		significance of compatibility rather than change.}
\end{flushleft}

\begin{flushleft}
	In the world of software management there exists a dreaded place called ``dependency hell.''
	The bigger your system grows and the more packages you integrate into your software, the
	more likely you are to find yourself, one day, in this pit of despair.
\end{flushleft}

\begin{flushleft}
	In systems with many dependencies, releasing new package versions can quickly become a nightmare.
	If the dependency specifications are too tight, you are in danger of version lock (the inability
	to upgrade a package without having to release new versions of every dependent package). If
	dependencies are specified too loosely, you will inevitably be bitten by version promiscuity
	(assuming compatibility with more future versions than is reasonable). Dependency hell is where
	you are when version lock and/or version promiscuity prevent you from easily and safely moving
	your project forward.
\end{flushleft}

\begin{flushleft}
	As a solution to this problem, we propose a simple set of rules and requirements that dictate
	how version numbers are assigned and incremented. These rules are based on but not necessarily
	limited to pre-existing widespread common practices in use in both closed and open-source software.
	For this system to work, you first need to declare a public API. This may consist of documentation
	or be enforced by the code itself. Regardless, it is important that this API be clear and precise.
	Once you identify your public API, you communicate changes to it with specific increments to your
	version number. Consider a version format of X.Y.Z (Major.Minor.Patch). Bug fixes not affecting
	the API increment the patch version, backwards compatible API additions/changes increment the minor
	version, and backwards incompatible API changes increment the major version.
\end{flushleft}

\begin{flushleft}
	We call this system ``Semantic Versioning.'' Under this scheme, version numbers and the way they
	change convey meaning about the underlying code and what has been modified from one version to the
	next.
\end{flushleft}

\begin{flushleft}
	Given a version number \icli{MAJOR.MINOR.PATCH}, increment the:
\end{flushleft}

\begin{enumerate}
	\item \icli{MAJOR} version when you make incompatible API changes,
	\item \icli{MINOR} version when you add functionality in a backwards compatible manner, and
	\item \icli{PATCH} version when you make backwards compatible bug fixes.
\end{enumerate}

\begin{flushleft}
	Additional labels for pre-release and build metadata are available as extensions to the
	\icli{MAJOR.MINOR.PATCH} format.
\end{flushleft}
