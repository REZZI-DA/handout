\section{Miscellaneous Commands}\label{git-misc-commands}

% === git remote

\subsection{Remotes}\label{git-remote}

\begin{flushleft}
	Change the remote URL on a local repository (here: \icli{origin}):
\end{flushleft}

\begin{flushleft}
	\cli{git remote set-url origin <url>}
\end{flushleft}

\begin{flushleft}
	Remove a remote:
\end{flushleft}

\begin{flushleft}
	\cli{git remote rm <remote>}
\end{flushleft}

% === git branch

\subsection{Branches}\label{git-branch}

\begin{flushleft}
	Rename the active branch:
\end{flushleft}

\begin{flushleft}
	\cli{git branch -m <newname>}
\end{flushleft}

\begin{flushleft}
	If you want to rename this branch on GitHub as well, run additionally:
\end{flushleft}

\begin{flushleft}
	\cli{git push origin -u <newname>}
\end{flushleft}

\begin{flushleft}
	To check out a branch from GitHub, use:
\end{flushleft}

\begin{flushleft}
	\cli{git fetch \flag{all}}
\end{flushleft}
\vspace{-0.4cm}
\begin{flushleft}
	\cli{git checkout -b <branch> origin/<branch>}
\end{flushleft}

\begin{flushleft}
	Delete a local and remote branch:
\end{flushleft}

\begin{flushleft}
	\cli{git branch \flag{delete} <branch>}
\end{flushleft}
\vspace{-0.4cm}
\begin{flushleft}
	\cli{git push origin \flag{delete} <branch>}
\end{flushleft}

% === git tag

\subsection{Tags}\label{git-tag}

\begin{flushleft}
	Add a new tag:
\end{flushleft}

\begin{flushleft}
	\cli{git tag <version>}
\end{flushleft}

\begin{flushleft}
	Add a new annotated tag:
\end{flushleft}

\begin{flushleft}
	\cli{git tag -a <version> -m <msg>}
\end{flushleft}

\begin{flushleft}
	Add a new annotated tag to a previous commit:
\end{flushleft}

\begin{flushleft}
	\cli{git tag -a <version> -m <msg> <sha>}
\end{flushleft}

\begin{flushleft}
	Push a specific tag to remote:
\end{flushleft}

\begin{flushleft}
	\cli{git push origin <version>}
\end{flushleft}

\begin{flushleft}
	Push all tags to remote:
\end{flushleft}

\begin{flushleft}
	\cli{git push origin \flag{tags}}
\end{flushleft}

\begin{flushleft}
	Delete a local tag:
\end{flushleft}

\begin{flushleft}
	\cli{git tag \flag{delete} <version>}
\end{flushleft}

\begin{flushleft}
	Delete a remote tag:
\end{flushleft}

\begin{flushleft}
	\cli{git push \flag{delete} origin <version>}
\end{flushleft}

% === git log

\subsection{Logs}\label{git-log}

\begin{flushleft}
	Display all commits from a particular author:
\end{flushleft}

\begin{flushleft}
	\cli{git log \flag{author}=<name> \flag{oneline}}
\end{flushleft}

\begin{flushleft}
	Get commits in a time range using Ruby expressions:
\end{flushleft}

\begin{flushleft}
	\cli{git log \flag{after} 10.days.ago \flag{oneline}}
\end{flushleft}

\begin{flushleft}
	View commit history as ASCII graph:
\end{flushleft}

\begin{flushleft}
	\cli{git log \flag{graph}}
\end{flushleft}

\begin{flushleft}
	Format log output:
\end{flushleft}

\begin{flushleft}
	\cli{git log \flag{pretty}:"<options>"}
\end{flushleft}

\begin{flushleft}
	See also this \url{https://git-scm.com/docs/pretty-formats} for more information.
\end{flushleft}

% === git diff

\subsection{Diffs}\label{git-diffs}

\begin{flushleft}
	Inspect diff statistics:
\end{flushleft}

\begin{flushleft}
	\cli{git diff \flag{stat}}
\end{flushleft}

\begin{flushleft}
	View the diff stats from last month:
\end{flushleft}

\begin{flushleft}
	\cli{git diff \flag{numstat} "\@{1 month ago}"}
\end{flushleft}

\begin{flushleft}
	View diffs from staged files; note that without this last flag nothing would be displayed:
\end{flushleft}

\begin{flushleft}
	\cli{git diff \flag{staged}}
\end{flushleft}

\begin{flushleft}
	View diffs between two branches:
\end{flushleft}

\begin{flushleft}
	\cli{git diff [\flag{name-only}] master...<branch>}
\end{flushleft}

\begin{enumerate}
	\item the \icli{\flag{name-only}} options only lists modified files without its diffs
	\item this syntax also works for comparing tags (with branches or other tags)
	\item the order of branches matters: \icli{<target-branch>...<active-branch>}
	\item the command \icli{git diff ..master} compares the checked-out branch to \icli{master}
\end{enumerate}

\begin{flushleft}
	You can refine this diffs even more with
\end{flushleft}

\begin{flushleft}
	\cli{git diff ..master {-}{-} <path> }
\end{flushleft}

% === git submodule

\subsection{Submodules}\label{git-submodule}

\begin{flushleft}
	Add a new submodule:
\end{flushleft}

\begin{flushleft}
	\cli{git submodule add <url>}
\end{flushleft}

\begin{flushleft}
	Update all submodules (after a pull):
\end{flushleft}

\begin{flushleft}
	\cli{git submodule update \flag{init} \flag{recursive}}
\end{flushleft}

\begin{flushleft}
	Submodules are removed the same way you would remove any file in Git.
\end{flushleft}

% === git stash

\subsection{Stash}\label{git-stash}

\begin{flushleft}
	This command is very useful if you want to change branches without having to
	commit or discard any changes made locally. See also the example at the end of
	this subsection for a more basic rundown and hands-on example on how to use this
	command.
\end{flushleft}

\begin{flushleft}
	\cli{git stash [\flag{include-untracked}|\flag{all}|\flag{path} <file(s)>]}
\end{flushleft}

\begin{flushleft}
	By default, this command ignores untracked or ignored files. If you don't want
	that, make sure to add the \icli{\flag{include-untracked}} option to also include
	untracked files as well (or use the  \icli{\flag{all}} option if you want to stash
	all untracked \emph{and} ignored files). To stash specific files, use the \icli{\flag{patch}}
	option which accepts one or more file paths.
\end{flushleft}

\begin{flushleft}
	You can review your stash with the
\end{flushleft}

\begin{flushleft}
	\cli{git stash list}
\end{flushleft}

\begin{flushleft}
	command which are stored using a LIFO strategy. By default, stashes are marked as
	WIP on top of the branch and commit that you created the stash from. However, this
	limited amount of information isn't helpful when you have multiple stashes, as it
	becomes difficult to remember or individually check their contents. To add a description
	to the stash, you can use the command
\end{flushleft}

\begin{flushleft}
	\cli{git stash save <description>}
\end{flushleft}

\begin{flushleft}
	As the name implies, this command applies the topmost stash (if no ID is specified):
\end{flushleft}

\begin{flushleft}
	\cli{git stash apply}
\end{flushleft}

\begin{flushleft}
	You can target a specific stash using stash IDs. The above command, for instance,
	is equivalent to
\end{flushleft}

\begin{flushleft}
	\cli{git stash apply@\{0\}}
\end{flushleft}

\begin{flushleft}
	The \icli{pop} command is similar \icli{apply}, but does one more thing in sequence:
	after applying the patch it drops this stash from the list. However, if there are
	conflicts when a stash is popped, \icli{pop} will not remove the stash.
\end{flushleft}

\begin{flushleft}
	\cli{git stash pop}
\end{flushleft}

\begin{flushleft}
	If you have multiple stashes, you can inspect the diff with the \icli{show} command.
	The \icli{\flag{path}} flag is optional here and produces a more verbose output.
\end{flushleft}

\begin{flushleft}
	\cli{git stash show stash@\{0\} [\flag{patch}]}
\end{flushleft}

\begin{flushleft}
	You can pass a stash ID to delete an individual stash from the stash list. For
	example, to remove the second stash run:
\end{flushleft}

\begin{flushleft}
	\cli{git stash drop@\{1\}}
\end{flushleft}

\begin{flushleft}
	To clear the entire stash, use:
\end{flushleft}

\begin{flushleft}
	\cli{git stash clear}
\end{flushleft}

\subsubsection{Switching Branches on the Fly}

\begin{flushleft}
	In this scenario, you are currently on the \icli{feature-01} branch, but want
	to check out the \icli{feature-02} branch with a untracked changes in your
	current directory:
\end{flushleft}

\begin{flushleft}
	\cli{git stash \flag{all}}
\end{flushleft}
\vspace{-0.6cm}
\begin{flushleft}
	\cli{git checkout feature-02}
\end{flushleft}
\vspace{-0.6cm}
\begin{flushleft}
	\cli{\# taking a quick gander here :)}
\end{flushleft}
\vspace{-0.6cm}
\begin{flushleft}
	\cli{git checkout feature-01}
\end{flushleft}
\vspace{-0.6cm}
\begin{flushleft}
	\cli{git stash pop}
\end{flushleft}

% === git clean

\subsection{Clean}\label{git-clean}

\begin{flushleft}
	Remove all untracked files. If the Git variable \icli{clean.requireForce} is
	not set to \icli{false}, \icli{clean} will refuse to delete files or directories
	unless a second \icli{-f} is given.
\end{flushleft}

\begin{flushleft}
	\cli{git clean <-i|-n|-f> -d .}
\end{flushleft}

\begin{itemize}
	\item \icli{-i} (or \icli{\flag{interactive}}) will show what would be done and clean files interactively
	\item \icli{-n} (or \icli{\flag{dry-run}}) will not remove anything and only show what would be done
	\item \icli{-f} (or \icli{\flag{force}}) will set \icli{clean.requireForce} to false
\end{itemize}
