\section{Configuration}\label{configuation}

\begin{flushleft}
	Git uses a series of configuration files to determine non-default behavior. There
	are three different configuration levels, each of which serves a different purpose.
	Note that higher-level configuration settings take precedence over their children.
\end{flushleft}

\begin{enumerate}
	\item \icli{\flag{system}} controls the settings for all users and all repositories on your computer
	\item \icli{\flag{global}} controls the settings for the currently logged-in user and all his repositories
	\item \icli{\flag{local}} controls the settings for the repository it resides in
\end{enumerate}

\begin{flushleft}
	One of the things you want to set first is an username and an email address
	to help Git identify you as the author. Most of the time, it is recommended
	to store these settings in the global configuration file.
\end{flushleft}

\begin{flushleft}
	\cli{git config \flag{global} user.name <username>}
\end{flushleft}
\vspace{-0.6cm}
\begin{flushleft}
	\cli{git config \flag{global} user.email <email>}
\end{flushleft}

\begin{flushleft}
	If you configure a new account on a machine where Git hasn't been used before,
	make sure to set master as the default branch name everywhere.
\end{flushleft}

\begin{flushleft}
	\cli{git config \flag{global} init.defaultBranch master}
\end{flushleft}

\begin{flushleft}
	Other than that there are a few more settings you may want to tweak, though there
	are much more user-specific because they depend on the environment you're operating
	in. The core editor is the editor Git will use when you want to write a multiline
	commit message. For example, if VS Code is your editor of choice you would want to write:
\end{flushleft}

\begin{flushleft}
	\cli{git config \flag{global} core.editor "code \flag{new-window} \flag{wait}"}
\end{flushleft}

\begin{flushleft}
	Another important configuration you want to make helps you avoid line-ending
	issues if you work on cross-platform projects. Even if you're not involved in
	cross-platform projects at the moment, it's better to settle this matter now
	before any problems start to surface later down the line.
\end{flushleft}

\begin{flushleft}
	Windows uses both a carriage-return character (\icli{\carret}) and a linefeed character
	(\icli{\linefeed}) for newlines in its files (CRLF), whereas macOS and Linux systems only
	use the linefeed character (LF). Git can handle this by auto-converting CRLF line endings into LF
	when you add a file to the index, and vice versa when it checks out code onto your filesystem. You
	can turn on this functionality with the \icli{core.autocrlf} setting. If you’re on a Windows machine,
	set it to \icli{true} -- this converts LF endings into CRLF when you check out code:
\end{flushleft}

\begin{flushleft}
	\cli{git config \flag{global} core.autocrlf true}
\end{flushleft}

\begin{flushleft}
	If you're on a Linux or macOS system that uses LF line endings, then you don’t want Git to automatically
	convert them when you check out files; however, if a file with CRLF endings accidentally gets introduced,
	then you may want Git to fix it. You can tell Git to convert CRLF to LF on commit but not the other way
	around by setting \icli{core.autocrlf} to \icli{input}:
\end{flushleft}

\begin{flushleft}
	\cli{git config \flag{global} core.autocrlf input}
\end{flushleft}

\begin{flushleft}
	This setup should leave you with CRLF endings in Windows checkouts, but LF endings on macOS and Linux
	systems and in the repository. If you're a Windows programmer doing a Windows-only project, then you
	can turn off this functionality, recording the carriage returns in the repository by setting \icli{core.autocrlf}
	to \icli{false}:
\end{flushleft}

\begin{flushleft}
	\cli{git config \flag{global} core.autocrlf false}
\end{flushleft}

\subsection{Alias}

\begin{flushleft}
	One of the major advantages of using Git from the terminal is that you can easily
	define your own commands that suits your needs best. See below a list of aliases
	I frequently use in my day-to-day work.
\end{flushleft}

\begin{flushleft}
	Unstage a file without discarding any changes:
\end{flushleft}

\begin{flushleft}
	\cli{git config \flag{global} alias.unstage 'reset HEAD {-}{-}'}
\end{flushleft}

\begin{flushleft}
	Uncommit a file without discarding any changes:
\end{flushleft}

\begin{flushleft}
	\cli{git config \flag{global} alias.uncommit 'reset \flag{soft} HEAD\^{}1'}
\end{flushleft}

\begin{flushleft}
	Show the last commit:
\end{flushleft}

\begin{flushleft}
	\cli{git config \flag{global} alias.last 'log -1 HEAD'}
\end{flushleft}

\begin{flushleft}
	Display log history as graph. Can be extended through the \icli{\flag{branches} \flag{tags}}
	and \icli{\flag{all}} flags, making the output of this command increasingly verbose:
\end{flushleft}

\begin{flushleft}
	\hbox{\cli{git config \flag{global} alias.graph 'log \flag{graph} \flag{oneline} \flag{decorate}'}}
\end{flushleft}

\subsection{Dotfiles Repository}\label{dotfiles-repository}

\begin{flushleft}
	A dotfile repository is a special repository Linux user use to store configuration
	files.
\end{flushleft}

\begin{flushleft}
	Create a new dotfiles repository:
\end{flushleft}

\begin{flushleft}
	\cli{mkdir -p \home/documents/repos}
\end{flushleft}
\vspace{-0.6cm}
\begin{flushleft}
	\cli{git init \flag{bare} \home/documents/dotfiles}
\end{flushleft}

\begin{flushleft}
	Then add this line to \icli{\home/.bashrc}:
\end{flushleft}

\begin{flushleft}
	\hbox{\cli{alias config="/usr/bin/git \flag{git-dir}=\home/documents/repos/dotfiles \flag{work-tree}=\$HOME"}}
\end{flushleft}

\begin{flushleft}
	Next run
\end{flushleft}

\begin{flushleft}
	\cli{config config \flag{local} status.showUntrackedFiles no}
\end{flushleft}

\begin{flushleft}
	Now, to add new configuration files, use the \icli{config} alias:
\end{flushleft}

\begin{flushleft}
	\cli{config add <file>}
\end{flushleft}
\vspace{-0.6cm}
\begin{flushleft}
	\cli{config commit -m <msg>}
\end{flushleft}
\vspace{-0.6cm}
\begin{flushleft}
	\cli{config push}
\end{flushleft}

\begin{flushleft}
	To access your dotfiles from a new machine, run through the following steps:
\end{flushleft}

\begin{flushleft}
	\cli{cd \home/documents/repos}
\end{flushleft}
\vspace{-0.6cm}
\begin{flushleft}
	\cli{git clone \flag{bare} <url> ./dotfiles}
\end{flushleft}

\begin{flushleft}
	Define the \icli{config} alias temporarily in your current shell session:
\end{flushleft}

\begin{flushleft}
	\hbox{\cli{alias config="/usr/bin/git \flag{git-dir}=\home/documents/repos/dotfiles \flag{work-tree}=\$HOME"}}
\end{flushleft}

\begin{flushleft}
	Hide all untracked files on your home director:
\end{flushleft}

\begin{flushleft}
	\cli{cd \home{} \&\& config config \flag{local} status.showUntrackedFiles no}
\end{flushleft}

\begin{flushleft}
	Removes	your local \icli{\home/.bashrc} and \icli{\home/.bash\_profile} files.
	If this is not what you want, rename and/or backup these files. This step is
	mandatory to checkout the branch on the dotfiles repository.
\end{flushleft}

\begin{flushleft}
	\cli{rm .bashrc .bash\_profile}
\end{flushleft}
\vspace{-0.6cm}
\begin{flushleft}
	\cli{config checkout}
\end{flushleft}
\vspace{-0.6cm}
\begin{flushleft}
	\cli{source \home/.bashrc}
\end{flushleft}
