\section{Bash}\label{bash}

\begin{flushleft}
	\emph{Bash is the GNU Project's shell and short for the Bourne Again SHell.
		As of today, Bash is the most popular shell among Linux users. Due to its
		popularity and widespread usage, Bash has also been ported to Windows and
		distributed with \href{https://cygwin.com/}{Cygwin} and
		\href{https://osdn.net/projects/mingw/}{MinGW}. As opposed to Windows,
		UNIX is an entire ecosystem self-tuned around text files.}
\end{flushleft}

\subsection{Profile Setup}\label{bash-profile}

\begin{flushleft}
	On most systems, \icli{\home/.bashrc} is only used when you startup an interactive
	non-login shell. However, if you open a new shell it is often an interactive login
	shell which completely ignores your configuration in \icli{\home/.bashrc} unless
	source the \icli{\home/.bashrc} from your \icli{\home/.profile} or \icli{\home/.bash\_profile}
	file:
\end{flushleft}

\begin{flushleft}
	\cli{[[ -f \home/.bashrc ]] \&\& .\, \home/.bashrc}
\end{flushleft}

\subsection{Core Commands}\label{bash-core-commands}

\begin{flushleft}
	Create a new file:
\end{flushleft}

\begin{flushleft}
	\cli{touch <file>}
\end{flushleft}

\begin{flushleft}
	Initialize a \icli{txt} file with a string value:
\end{flushleft}

\begin{flushleft}
	\cli{echo "Hello, World!" > test.txt}
\end{flushleft}

\begin{flushleft}
	Print out the content of \icli{test.txt} to standard output:
\end{flushleft}

\begin{flushleft}
	\cli{cat [\flag{number}] test.txt}
\end{flushleft}

\begin{flushleft}
	Print the first ten lines of the file. Pipe the output to \icli{tail} to
	reverse this effect:
\end{flushleft}

\begin{flushleft}
	\cli{cat test.txt | head}
\end{flushleft}

\begin{flushleft}
	Create a new directory. Use the \icli{-p} if one or more parents don't exist.
\end{flushleft}

\begin{flushleft}
	\cli{mkdir [-p] <path>}
\end{flushleft}

\begin{flushleft}
	Move a file to a new location. This Cmdlet can also be used to rename a file:
\end{flushleft}

\begin{flushleft}
	\cli{mv <path> <destination>}
\end{flushleft}

\begin{flushleft}
	List all files in a specific directory. Use the \icli{g} flag for a long listing
	format (excluding the owner), and \icli{\flag{color}=auto} for colorful output.
\end{flushleft}

\begin{flushleft}
	\cli{ls <dir> [-g|\flag{color=auto}]}
\end{flushleft}

\begin{flushleft}
	List all \icli{tex} files from the \icli{src} directory of this document,
	and format the output accordingly.
\end{flushleft}

\begin{flushleft}
	\cli{ls -g {-}{-}  src/**/*.tex}
\end{flushleft}

\begin{flushleft}
	Remove a file:
\end{flushleft}

\begin{flushleft}
	\cli{rm <file>}
\end{flushleft}

\begin{flushleft}
	Remove an empty directory:
\end{flushleft}

\begin{flushleft}
	\cli{rmdir <dir>}
\end{flushleft}

\begin{flushleft}
	Remove a non-empty directory recursively and run this command in a dry-run mode:
\end{flushleft}

\begin{flushleft}
	\cli{rm -rf <dir> \flag{dry-run}}
\end{flushleft}
